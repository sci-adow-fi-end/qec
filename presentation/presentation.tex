%allocazione tempo 
%
%2 minuti "a cosa servono"
%2 minuti "qubit e porte+ full adder e sqrt not"
%2 minuti "interferenza e entanglement + errori quantistici + confronto computazione analogica"
%2 minuti "error correction e teorema della soglia"
%




%uno speedup polinomiale non rompe la barriera della trattabilità 
%o intrattabilità dei problemi 


\documentclass{beamer}
\usepackage[utf8]{inputenc}
\usepackage{graphicx}
\usepackage{amssymb}
\usepackage{amsfonts}
\usepackage{amsmath}
\usepackage{braket}
\usepackage{tikz}
\usepackage{float}
\usepackage{hyperref}
\usepackage{subcaption}
\usetikzlibrary{quantikz2}
\usepackage{pgfplots}
\usetheme{Copenhagen}
\usecolortheme{beaver}

\title{Quantum errors and error corretion tecniques}

\author{Alessio Delli Colli}

\date{September 2024}

\begin{document}

\maketitle

%i numeri sono tutti figli legittimi della matematica
%l'algoritmo di Shor o sue opportune varianti potrebbero risolvere 
%il problema della Fattorizzazione e il problema del lograritmo discreto anche su 
%curve ellittiche. 
%potrebbero e non possono perchè l'hardware disponibile al momento non è ancora adatto 
%a questo tipo di computazione. 
%il numero più alto che si è riusciti a fattorizzare con l'algoritmo di Shor è 21,
%anche il 35 è stato tentato ma senza successo perche la propagazione degli 
%errori ha reso il risultato inutilizzabile
%

\begin{frame}
	\frametitle{The need for quantum computation}
	\textbf{It grants speedup for "difficult" computational problems}

	\pause
	\vspace{30pt}
	\begin{itemize}
		\item Search problems with Grover's algorithm
		      \vspace{30pt}
		      \pause
		\item Factorization and discrete logarithm with Shor's algorithm.


	\end{itemize}
	\vspace{20pt}
	\pause
	But there is a catch...
	\pause
	Errors
\end{frame}


%parla del nome circuito
%descrizione del 
\begin{frame}
	\frametitle{Qubits}
	\begin{itemize}
		\item simple quantum systems
		      \pause
		\item modeled by a 2-dimensional Hilbert space
		      \pause
		\item their state can be represented in many ways:
		      \pause
	\end{itemize}

	\vspace{10pt}

	\noindent
	\begin{minipage}{0.5\textwidth}
		\begin{center}
			\textbf{state vector}
		\end{center}
		\begin{equation*}
			\alpha \ket{0} + \beta \ket{1}
		\end{equation*}
	\end{minipage}
	\hfill
	\pause
	\begin{minipage}{0.48\textwidth}
		\begin{center}
			\textbf{Bloch sphere}
		\end{center}
		Right column content goes here.
		sdvbskdvbksdbvksdbvkjdsbvsdkfbv
	\end{minipage}
\end{frame}

\begin{frame}
	\frametitle{Porte e circuiti quantistici}

	\begin{minipage}{0.31\textwidth}
		\begin{center}
			\textbf{Bloch sphere}
		\end{center}
		Right column content goes here.
		sdvbskdvbksdbvksdbvkjdsbvsdkfbv
	\end{minipage}	\begin{minipage}{0.31\textwidth}
		\begin{center}
			\textbf{Bloch sphere}
		\end{center}
		Right column content goes here.
		sdvbskdvbksdbvksdbvkjdsbvsdkfbv
	\end{minipage}	\begin{minipage}{0.31\textwidth}
		\begin{center}
			\textbf{Bloch sphere}
		\end{center}
		Right column content goes here.
		sdvbskdvbksdbvksdbvkjdsbvsdkfbv
	\end{minipage}
\end{frame}

%nonostante il sistema sia inesorabilmente legato con il mondo esterno noi vogliamo 
%perdere questo legame all'interno della nostra rappresentazione. 
%perche il ondo esterno non lo possiamo misurare, non lo possiamo influenzare in 
%maniera apprezzabile. 

\begin{frame}
	\frametitle{Quantum errors}
\end{frame}

%il mondo dei qubits continuo assomiglia molto di piu al mondo discreto dei bits 
%che al mondo continuo dei qubit 

\begin{frame}
	\frametitle{Classical error correction}
\end{frame}

\begin{frame}
	\frametitle{Challenges in the quantum case}
\end{frame}

\begin{frame}
	\frametitle{Quantum codes}
\end{frame}

\begin{frame}

\end{frame}
\end{document}
